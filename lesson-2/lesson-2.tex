\documentclass[10pt,a4paper]{article}

\usepackage[OT1]{fontenc}
\usepackage{amsmath}
\usepackage{amsfonts}
\usepackage{amssymb}
%\usepackage{lmodern}%
\usepackage[left=1cm,right=1cm,top=2cm,bottom=2cm]{geometry}
\usepackage[T1,T2A]{fontenc}
\usepackage[lutf8]{luainputenc}
\usepackage[english,russian]{babel}

\title{Практическое задание к лекции №2}
\date{15.10.2021}
\begin{document}
\maketitle

\section{Представьте в виде несократимой рациональной дроби:}
\begin{enumerate}
\item[a)] $0.(216)$ \\[6pt]
$$x = 0.(216)$$
$$1000x = 216 + x$$
$$x=\frac{216}{999} = \boxed{\frac{8}{37}}$$

\item[б)] $1.0(01)$ \\[6pt]
$k=$ количество цифр в периоде $= 2$ \\
$m=$ длина дробной части до начала периода $= 1$ \\
$y=$ целая часть числа $= 1$ \\
$a=$ число, составленное из цифр дробной части $= 1$ \\
$b=$ число, составленное из цифр дробной части до начала периода $= 0$ \\[10pt]

\begin{center}
По формуле $y + \dfrac{a-b}{9[k \: times]0[m \: times]}$ \\[10pt]
получаем: $1 + \dfrac{1}{990}= \boxed{1\dfrac{1}{990}}$
\end{center}
\end{enumerate}

\section{Проверьте любым способом, являются ли данные логические формулы
тавтологией:}

\begin{enumerate}
\item[a)] $(A \lor B)\implies (B \lor \neg A)$ \\[10pt]
Проверим выражение на тавтологию с помощью поиска контрпримера (это быстрее построения полной таблицы истинности):
\begin{center}
Импликация = 0, только если $(A \lor B)=1$ и $(B \lor \neg A)=0$ \\[6pt]
Выражение $(B \lor \neg A)=0$, только если $B=0$ и $A=1$\\[6pt]
Подставив значения $B=0$ и $A=1$ в выражение $(A \lor B)=1$, получаем $(1 \lor 0) = 1$\\[6pt]
Контрпример найден $\therefore$ \boxed{$исходное выражение -- не тавтология$}

\end{center}

\item[б)] $A \implies (A \lor (\neg B \land A))  $ \\[10pt]
Проверим выражение на тавтологию с помощью поиска контрпримера:
\begin{center}
Импликация = 0, только если A=1 и $(A \lor (\neg B \land A))=0$ \\[6pt]
Подставив значение $A$ во второе выражение, получаем $(1 \lor (\neg B \land 1))=0$ \\[6pt]
Выражение $(1 \lor (\neg B \land 1)) \neq 0$ при любом $B$, т.к. первый член этой дизъюнкции всегда равен 1. \\[6pt]
Контрпример не найден $\therefore$ \boxed{$исходное выражение -- тавтология$}

\end{center}

\end{enumerate}

\section{Сформулируйте словесно высказывания:} 

\begin{enumerate}
\item[a)] $(\neg A \lor B)\implies \neg C$ \\[10pt]
\boxed{$Если сегодня не светит солнце или сыро, то я не поеду на дачу.$} \\[6pt]

\item[б)] $C \implies (A \lor \neg B)$ \\[10pt]
\boxed{$Если я поеду на дачу, значит сегодня светит солнце или не сыро.$}

\end{enumerate}

\paragraph{где A: сегодня светит солнце; В: сегодня сыро; С: я поеду на дачу.} 

\end{document}