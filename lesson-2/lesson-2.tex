\documentclass[10pt,a4paper]{article}

\usepackage[OT1]{fontenc}
\usepackage{amsmath}
\usepackage{amsfonts}
\usepackage{amssymb}
\usepackage{lmodern}
\usepackage[left=1cm,right=1cm,top=2cm,bottom=2cm]{geometry}

\usepackage[T1,T2A]{fontenc}
\usepackage[lutf8]{luainputenc}
\usepackage[english,russian]{babel}


\title{Практическое задание к лекции №2}
\date{15.10.2021}

\begin{document}
\maketitle

\section{Представьте в виде несократимой рациональной дроби:}
\begin{enumerate}
\item[a)] $0.(216)$
$$x = 0.(216)$$
$$1000x = 216 + x$$
$$x=\frac{216}{999} = \boxed{\frac{8}{37}}$$


\item[б)] $1.0(01)$ \\
\begin{center}
$k=$ количество цифр в периоде $= 2$ \\
$m=$ длина дробной части до начала периода $= 1$ \\
$y=$ целая часть числа $= 1$ \\
$a=$ число, составленное из цифр дробной части $= 1$ \\
$b=$ число, составленное из цифр дробной части до начала периода $= 0$ \\

По формуле $y + \dfrac{a-b}{9[k \: times]0[m \: times]}$ \\
получаем: $1 + \dfrac{1}{990}= \boxed{1\frac{1}{990}}$


\end{center}

\end{enumerate}

\section{Проверьте любым способом, являются ли данные логические формулы
тавтологией:}

\section{Сформулируйте словесно высказывания:}





\end{document}